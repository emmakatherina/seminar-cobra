\documentclass{article}

\usepackage[utf8]{inputenc}
\usepackage{ngerman}
\usepackage{lmodern}
\usepackage{amsthm}
\usepackage{amssymb}

\newtheorem{satz}{Satz}
\newtheorem{definition}[satz]{Definition}
%Create a new theorem by writing \newtheorem{How to call this type of theorem}[satz]{What should be written}. Make sure to keep "satz" to ensure consecutive numeration.

\title{Dimension von Varietäten}
\date{\today}
\author{Yvan was-ist-dein-Nachname \and Emma Ahrens}

\begin{document}

\maketitle
\tableofcontents

\section{Abstract}
\section{Einleitung}
\section{Dimension von Monomidealen}

	\begin{satz}
	Sei \(\{0\} \neq I \subseteq k[X_{1},\ldots,X_{n}]\) ein Ideal und \(\preceq\) eine Monomialordnung auf \(Z^{n}_{\geq 0}\). Sei G eine Gröbnerbasis von I mit I = (G). Dann ist eine k-Basis von \(k[X_{1},\ldots,X_{n}]/I\) gegeben durch die Restklassen von \(X^{\alpha}\) mit
	\begin{displaymath}
	\alpha \in C(I) := \{\alpha \in Z^{n}_{\geq 0}\, |\; LT(g) \nmid X^{\alpha}\quad \forall g \in G\}.
	\end{displaymath}
	\end{satz}
	

\section{Dimension von beliebigen Idealen}
\section{Literaturangabe}

\end{document}
