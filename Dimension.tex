\documentclass{article}

\usepackage[utf8]{inputenc}
\usepackage{ngerman}
\usepackage{lmodern}
\usepackage{amsthm}
\usepackage{amssymb}
\usepackage{mathtools}
\usepackage[paper=a4paper,left=25mm,right=25mm,top=25mm]{geometry}

\newtheorem{satz}{Satz}
\newtheorem{definition}[satz]{Definition}
\newtheorem{lemma}[satz]{Lemma}
\newtheorem{proposition}[satz]{Proposition}
\newtheorem{beweis}{Beweis}
%Create a new theorem by writing \newtheorem{How to call this type of theorem}[satz]{What should be written}. Make sure to keep "satz" to ensure consecutive numeration.
% %  Variablen Definition...............
\newcommand*{\R}{k[X_{1},\ldots,X_{n}]}
\newcommand*{\indx}[2]{{#1}_{#2}}
\newcommand*{\potx}[2]{{#1}^{#2}}
\newcommand*{\N}{\mathrm{N}_0}
\newcommand*{\hf}[1]{$\prescript{a}{}{HF}_{#1}$}
\newcommand*{\hp}[1]{$\prescript{a}{}{HP}_{#1}$}
\newcommand*{\kette}[2]{$1\leq {#1}_1<{#1}_2<{#1}_3<...<{#1}_{#2}\leq n$}
\newcommand*{\Rr}[2]{$ k[X_{{#1}_{1}},\ldots,X_{{#1}_{#2}}]$}
%\newcommand*{\hf}{}
\newcommand*{\ideal}{$I$}


\title{Dimension von Varietäten}
\date{\today}
\author{Yvan Ngumeteh \and Emma Ahrens}

\begin{document}

\maketitle
\tableofcontents

\section{Abstract}
\section{Einleitung}
\section{Dimension von Monomidealen}

	\begin{lemma} \label{1.2.3}
	Sei \(I \subseteq \R\) ein Ideal, das von einer Menge G von Monomen erzeugt wird. Dann liegt
	ein Polynom \(f \in \R\) in I genau dann, wenn für jeden Term \(a_{j}X^{\alpha_{j}}\) von f ein
	\(g \in G\) existiert, welches \(a_{j}X^{\alpha_{j}}\) teilt.
	\end{lemma}

	\begin{proof}[Beweis]
	Sei \(f \in I\). Dann gilt \(f = \sum_{i=1}^{s} h_{i}g_{i}\) mit \(h_{i} \in R\) und \(g_{i}
	\in G\). Damit hat jeder Term die Form \(h_{i}g_{i}\) und ist somit durch ein Element aus G
	teilbar.
	Sei nun andersherum \(f \in \R\) und für jeden Term \(a_{j}X^{\alpha_{j}}\) von f existiert ein
	\(g \in G\), welches \(a_{j}X^{\alpha_{j}}\) teilt. Dann kann man f als Linearkombination von Elementen aus G schreiben und damit liegt f nach der Definition eines Ideals in I.
	\end{proof}

	
	\begin{lemma} \label{1.2.4}
	Sei \((g_{i})_{i \geq 1}\) eine Folge von Monomen in \(\R\) mit \(g_{1} \succeq g_{2} \succeq
	\ldots\) für eine Monomialordnung \(\preceq\). Dann existiert ein \(r \in \mathbb{N}\) mit 
	\(g_{n} = g_{r}\) für alle \(n \geq r\). 
	\end{lemma}

	\begin{proof}[Beweis]
	Sei \(I = ((g_{i})_{i \geq 1})\), dann ist I ein Ideal. Nach dem Hilbert'schen Basissatz
	wissen wir, dass I endlich erzeugt ist. Also existiert ein r, so dass die Menge \(G = \{g_{1},
	\ldots, g_{r}\}\) I erzeugt. Für ein \(i \geq r\) und \(g_{i} \in I\) existiert ein
	\(j \in \underline{r}\), so dass \(g_{j}\; | \;(g_{i}\) nach Lemma~\ref{1.2.3}. Also \(g_{i}
	\succeq g_{j} \succeq g_{r}\). Andererseits gilt nach Voraussetzung, dass \(g_{i} \preceq g_{r}
	\), also folgt \(g_{i} = g_{r}\).
	\end{proof}

	Lemma~\ref{1.2.4} sagt uns, dass jede absteigende Kette von Monomen stationär wird und insbesondere in jeder abzählbaren Menge von Monomen ein kleinstes Element existiert.
	

	\begin{proposition}[Divisionsalgorithmus] \label{1.2.5}
	Sei \(\preceq\) eine Monomialordnung und \(f, f_{1}, \ldots, f_{s} \in \R\) nicht null. Dann
	gilt \begin{displaymath} f = \sum_{i=1}^{s} h_{i}f_{i}\; + r, \end{displaymath} mit
	\(r, h_{1}, \ldots, h_{s} \in \R\) und \(LT(h_{i}f_{i} \preceq LT(f)\) für alle \(h_{i} \neq 0
	\) und \(r = 0\) oder kein Term von r wird durch ein \(LT(f_{i})\) geteilt für \(i \in
	\underline{s}\).
	\end{proposition}

	\begin{proof}[Beweis]
	\end{proof}


	\begin{satz} \label{1.2.8}
	Sei \(\{0\} \neq I \subseteq \R\) ein Ideal und \(\preceq\) eine Monomialordnung auf
	\(Z^{n}_{\geq 0}\). Sei G eine Gröbnerbasis von I mit I = (G). Dann ist eine k-Basis von 
	\(\R/I\) gegeben durch die Restklassen von \(X^{\alpha}\) mit
	\begin{displaymath}
	\alpha \in C(I) := \{\alpha \in Z^{n}_{\geq 0}\, |\; LT(g) \nmid X^{\alpha}\quad \forall g 
	\in G\}.
	\end{displaymath}
	\end{satz}

	\begin{proof}[Beweis]
	Wir zeigen erst, dass die Monome mit Exponent aus C(I) ganz \(k[X_{1},\ldots,X_{n}]/I\) 
	aufspannen und anschließend, dass kein Element aus I durch echte Linearkombination solcher
	Monome dargestellt werden kann. \\
	Sei \(G = \{f_{1}, \ldots, f_{s}\}\) und \(0 \neq f \in \R\). Dann ist
	\(f = \sum_{i=1}^{s} h_{i}f_{i}\; + r = f' + r\) nach Proposition~\ref{1.2.5} mit \(r=0\) oder 
	\(r = a_{l}X^{\alpha_{l}} + \ldots + a_{0}\) mit \(LT(f_{i}) \nmid X^{\alpha_{j}}\) 
	für jedes \(i \in \underline{s}\) und \(j \in \underline{l}\). Also ist \(r\) eine
	Linearkombination von Monomen \(X^{\alpha_{j}}\) mit \(\alpha_{j} \in C(I)\).
	Es gilt außerdem \([f] = [r]\) in \(\R/(G)\) und damit erzeugen die
	Monome mit \(\alpha \in C(I)\) den ganzen Restklassenring. \\
	Angenommen es existiert \(f = f' + r \in I\) mit \(r \neq 0\) und f' und r wie oben.
	Dann gilt \(0 \neq r = f - f'\). Da \(f \in I\) und \(f' \in I\) folgt
	\(r \in I\), womit folgt, dass \((LT(r) \in (LT(f_{1}), \ldots, LT(f_{s}))\).
	Nach Lemma~\ref{1.2.3} existiert dann ein \(f_{i}\) mit \(LT(f_{i})\; |\; LT(r)\). Dies ist ein 
	Widerspruch, also folgt \(r = 0\) und die Restklassen von \(X^{\alpha}\) mit
	\(\alpha \in C(I)\) sind linear unabhängig in \(\R/I\).
	\end{proof}
	

\section{Dimension von beliebigen Idealen}

\subsection{Das Hilbert-Polynom}

Im folgenden sei $n \in \mathrm{N}$ fest.

\begin{satz}
	Sei  \ideal $\subset$ $\R$, dann existiert es einen eindeutigen Polynom \hp{I}(t) $\in \mathrm{Q}[t]$ (mit t eine variable) und $\indx{s}{0}\geq0$,  sodass \hp{I}(s)=\hf{I}(s)= $\indx{dim}{k}$ ($\indx{\R}{\leq s}$/$\indx{I}{\leq s}$), $\forall s\geq\indx{s}{0}$. Weiterhin besitzt \hp{I}(t) folgende Eigenschaften:	
\end{satz}
\begin{itemize}
	\item Der Grad von \hp{I}(t) ist der größte $d \in \mathrm{N}$, sodass es \kette{i}{d} existieren mit $I\cap k[X_{{i}_{1}},\ldots,X_{{i}_{d}}]={\emptyset}$.
	\item Sei $d=grad(\textnormal{\hp{I}(t)})$. Dann gilt \hp{I}(t)=$\sum_{k=0}^{d} \indx{a}{k}t^k$ mit $\indx{a}{k}d! \in \mathrm{Z}, \forall k\in \underline{\indx{d}{0}}$ und $\indx{a}{k}d!>0$
\end{itemize}

\begin{beweis}
	Wir bemerken dass \hp{I}(t) eindeutig ist, da es ein Polynom ist. Es nur die Existenz nachgewiesen werden. Sei $M=\{\alpha \in \N:|\alpha|\leq s\}$
	\begin{itemize}
		\item Für die trivialen Fällen $I=(0)$ hat man, wegen \hf{I}(s)= $\indx{dim}{k}$ ($\indx{\R}{\leq s}$/$\indx{I}{\leq s}$)$=$ $\left|M\right|=\binom{n+s}{s}, \forall s\in \N$.
		
		Oder $I=\R$ gilt \hf{I}(s)= $\indx{dim}{k}$ ($\indx{\R}{\leq s}$/$\indx{I}{\leq s}$)$=0, \forall s\in \N$ und somit entspricht in diesem Fall \hp{I}$=0$ (Das Nullpolynom !)
		Nehmen wir also an, dass $I$ nicht trivial ist. Sei G eine Gröbner-Basis von $I$ (bzgl. eine graduierte lexikographische Ordnung) und 
		\begin{displaymath}
		\{LM(g):g\in G\}=\{\indx{X}{\beta}:\beta \in M \}
		\end{displaymath}
		
		wir setzen
		\begin{displaymath}
		C(I):=\{\alpha \in \N: \potx{X}{\beta} \nmid \potx{X}{\alpha} \forall \in \beta \in M \}
		\textnormal{ und } 
		\indx{C(I)}{\leq s}:= C(I) \cap \{\alpha \in \N:|\alpha|\leq s\}
		\end{displaymath}
		
Behauptung: Für $s\geq 0$ gilt \hf{I}(s)$=|\indx{C(I)}{\leq s}|$, $\forall s\geq 0$.

Für den Beweis benutzt man (Macaulay), dann gilt \hf{I}(s)=\hf{(LT(I))}(s), $\forall \in s\geq0$. Das heißt,
\begin{displaymath}
\indx{dim}{k} (\indx{\R}{\leq s}/\indx{I}{\leq s})=\indx{dim}{k}(\indx{\R}{\leq s}/\indx{(LT(I))}{\leq s}).
\end{displaymath}

Weiterhin gilt mit der Buchberger-Definition (1.2.7), dass $\{\potx{X}{\beta}:\beta \in M \}$ ist eine Gröebner-Basis von (LT(I)), deshalb mit Satz 1.2.8 habt man, dass die Restklassen von $\potx{X}{\beta}$ ($\alpha\in C(I)$) bilden eine K-Vektorraum Basis von $\R$/(LT(I)). Daraus folgt die Behauptung.

\item Sei $J\subseteq \underline{n}$ und eine Funktion $\tau:J\longrightarrow \N$. Wir definieren
\begin{displaymath}
C(J,\tau):=\{\alpha \in N: \indx{\alpha}{j}=\tau(j), \forall \in J  \}
\end{displaymath}

Behauptung: Es existiert eine endliche Anzahl $\chi$ von Tupeln (J,$\tau$), sodass 

\begin{displaymath}
C(I)=\bigcup\limits_{(J,\tau)\in \chi}
\end{displaymath}

\begin{proof}
	Für $\beta:=(\indx{\beta}{1},\ldots,\indx{\beta}{n}) \in \N$, definiert man 
	
	\begin{displaymath}
	C(\beta):=\{\alpha\in \potx{\N}{n}: \potx{X}{\beta}\nmid\potx{X}{\alpha}\}
	\end{displaymath}
	Dann haben wir $C(I)=\bigcap\limits_{\beta\in M}C(\beta)$. 
	Weiterhin bemerken wir, dass falls $(J,\tau),(J\prime,\tau\prime)$ zwei Tupeln, wie oben definiert bezeichnet, dann gilt 
	
	\begin{displaymath}
	C(J,\tau)\cap C(J\prime,\tau\prime)=\left\{\begin{array}{ll} \emptyset, & falls  \tau(j)=\tau\prime(j) \\
	C(J\cup J\prime, \tau_0), & sonst \end{array}\right.
	\end{displaymath} 
	wobei $\indx{\tau}{0}:J\cup J\prime \longrightarrow \N$, $ j\mapsto \tau_{0}(j)=\left\{\begin{array}{ll} \tau(j), & falls  j\in J \\
	\tau\prime(j), & falls  j\in J\prime \\
	0, & sonst \end{array}\right.$.
	
	Das heißt man kann O.B.d.A annehmen, dass in 
 	
\end{proof}
	\end{itemize}
\end{beweis}
%\minisec{Behauptung}	




\begin{thebibliography}{99}
	\bibitem{Heu03}
	\textsc{Heuser}, Harro:
	\newblock \emph{Lehrbuch der Analysis}.
	\newblock 15. Aufl.
	\newblock Vieweg-Verlag, Braunschweig-Wiesbaden, 2003
	
	\bibitem{GM-GM}
	\textsc{Gr{\"o}ger}, Detlef ; \textsc{Marti}, Kurt:
	\newblock \emph{Grundkurs Mathematik für Ingenieure, Natur- und
		Wirtschaftswissenschaftler}.
	\newblock 2.~Aufl.
	\newblock Physica-Verlag, 2004
\end{thebibliography}


\end{document}
