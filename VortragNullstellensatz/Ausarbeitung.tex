\documentclass{article}

\usepackage[utf8]{inputenc}
\usepackage{ngerman}
\usepackage{lmodern}
\usepackage{amsthm}
\usepackage{amssymb}
\usepackage{amsmath}
\usepackage[paper=a4paper,left=25mm,right=25mm,top=25mm]{geometry}
\usepackage{mathtools}
\usepackage{graphicx}
\usepackage{paralist}
\usepackage{color}

\newtheorem{satz}{Satz}
\newtheorem{definition}[satz]{Definition}
\newtheorem{lemma}[satz]{Lemma}
\newtheorem{proposition}[satz]{Proposition}
\newtheorem{korollar}[satz]{Korollar}

\title{Hilbert's Nullstellensatz}
\date{\today}
\author{Yvan Ngumeteh \and Emma Ahrens}

\begin{document}

\maketitle
%\tableofcontents

\section{Abstract}
\section{Einleitung}
\section{Hyperebenen}
	
	\begin{satz}[Hilbert's Nullstellensatz für Hyperebenen]
	Sei k algebraisch abgeschlossen, \(f \in k[X_1,\ldots,X_n]\) nicht
	konstant und \(\emptyset \neq H_f \subseteq k^n\) die korrespondierende
	Hyperebene. Wir können f schreiben als \(f = f_1^{n_1}\cdots f_r^{n_r}\)
	mit \(f_1,\ldots,f_r\) irreduzibel und paarweise teilerfremd. Dann ist
	\begin{displaymath} H_f = H_{f_1} \cup \cdots \cup H_{f_r} \text{und }
	\mathbf{I}(H_f) = (f_1\cdots f_r). \end{displaymath}
	Insbesondere gilt, falls f irreduzibel ist, dass \(\mathbf{I}(H_f) = (f)\).
	\end{satz}

\section{Schwache Form}

	\begin{definition}[Algebraische Elemente]
	Sei A eine k-Algebra. Dann heißt die Menge \(a_1,\ldots, a_m \in A\)
	algebraisch unabhängig, falls kein Polynom \(0 \neq F \in k[X_1,\ldots,X_m]
	\) existiert mit \(F(a_1,\ldots,a_m) = 0\).
	\end{definition}

	Im Folgenden sind A und B kommmutative Ringe mit Eins und \(A \subseteq B\).

	\begin{definition}[Ganze Elemente]
	Wir nennen \(b \in B\) ganz über A, wenn es Elemente \(a_1, \ldots, a_n \in
	A\) gibt mit
	\begin{displaymath} b^n + a_{n-1}b^{n-1} + \cdots + a_1b + a_0 = 0
	\end{displaymath} für ein \(n \in \mathbb{N}\).
	Außerdem heißt B ganz über A, wenn jedes Element aus B ganz über A ist.
	\end{definition}

	\begin{lemma} \label{2.1.1}
	Sei \(b \in B\). Dann ist äquivalent:
	\begin{enumerate}
	\item b ist ganz über A
	\item Der von b erzeugte Teilring \(A[b] \subseteq B\) ist ein endlich
	erzeugter A-Modul.
	\item Es existiert ein Teilring \(C \subseteq B\) mit \(A[b] \subseteq C\)
	und C ist ein endlich erzeugter A-Modul.
	\end{enumerate}
	\end{lemma}

	\begin{proof}[Beweis]
	(\(1 \Rightarrow 2\)): Es ist \(A[b] = \{f(b)\;|\;f\in A[X]\}\)
	und da b ganz ist, existiert ein Polynom \(0 \neq g \in A[X]\) mit \(g(b) 
	= 0\) und \(Grad(g) = n \geq 1\). Da \(A[X]\) ein euklidischer Ring ist, können
	wir jedes \(f \in A[X]\) schreiben als \(f = qg + r\) mit \(q,r \in A[X]\)
	und \(Grad(r) < n\). Also \(f(b) = q(b)*g(b) + r(b) = r(b)\) und f ist eine
	A-Linearkombination von \(1, b, b^2, \ldots, b^{n-1}\), also ist \(A[b]\)
	endlich erzeugt. \\
	(\(2 \Rightarrow 3\)): Setze \(C := A[b]\), dann ist C ein Teilring von B
	und die Aussage folgt. \\
	(\(3 \Rightarrow 1\)): Seien \(c_1, \ldots, c_n \in C\) mit \(C =
	\sum_{i=1}^n Ac_i\). Es gilt \(b \in A[b] \subseteq C\), also auch \(bc_i
	\in C\) und es existieren die \(a_{ij} \in A\) mit \(bc_i = \sum_{j=1}^n
	a_{ij}c_i\). Sei \(M \in A^{n\times n}\) eine Matrix mit \((M)_{i,j} =
	a_{ij}\) für alle \(i, j \in \underline{n}\) und \(v \in A^n\) der Vektor
	mit \(v_i = c_i\) wie oben. Dann entsprechen die obigen Gleichungen dem
	Gleichungssystem
	\begin{displaymath}Mv = bv \Leftrightarrow (M-I_n)v = 0.\end{displaymath}
	Die Cramersche Regel besagt, dass \(v_i = \frac{Det((M-I_n)_i)}{Det(M-I_n)}
	\Leftrightarrow v_iDet(M-I_n) = Det((M-I_n)_i)\), wobei in die Matrix
	\((M-I_n)_i \) in unserem Fall nur Nullen in der i-ten Spalte stehen. Also
	gilt \begin{displaymath} Det(M-I_n)_i) = 0 \Rightarrow v_iDet(M-I_n) = 0.
	\end{displaymath}
	Wir müssen noch zeigen, dass daraus \(Det(M-I_n) = 0\) folgt, denn dann 
	können wir die Determinante ausschreiben und \(1, b, \ldots\) wird linear
	abhängig über A, also ist b ganz über A. \\
	Es ist \(1 \in C\), also existiert eine Linearkombination \(1 =
	\sum_{i=1}^n a_ic_i \Leftrightarrow Det(M-I_n) = \sum_{i=1}^n
	a_ic_iDet(M-I_n) = 0\). Also gilt \(Det(M-I_n) = 0\) und die
	Behauptung folgt.
	\end{proof}

	\begin{korollar} \label{2.1.2}
	Seien A, B kommutative Ringe mit \(A \subseteq B\).
	\begin{enumerate}
	\item\label{2.1.2 a} Falls \(B = A[b_1, \ldots, b_n]\), wobei jedes \(b_i
	\in B\) ganz über \(A[b_1, \ldots, b_{i-1}]\) ist, dann ist B endlich
	erzeugter A-Modul und ganz über A.
	\item Die Menge \(\overline{A}_B := \{b \in B\;|\;b \text{ ganz über } A\}\) ist
	ein Teilring von B und heißt ganzer Abschluss von A in B.
	\item Sei \(C \subseteq B\) ein Teilring mit \(A \subseteq C\). Falls C
	ganz ist über A und B ganz ist über C, dann ist auch B ganz über A.
	\item Falls B ein Körper ist und ganz über A, dann ist A auch ein Körper.
	\end{enumerate}
	\end{korollar}

	\begin{proof}[Beweis]
	\begin{enumerate}
	\item Beweis durch Induktion über \(n \in \mathbb{N}\).\\
	\(n = 1\): Sei \(B_n = B_1 = A[b_1]\) und \(b_1\) ganz über A. Dann folgt
	mit Lemma~\ref{2.1.1}, dass \(A[b_1]\) ein endlich erzeugter A-Modul ist
	und \(A[b_1]\) ganz über A ist. \\
	Angenommen die Behauptung gilt für ein beliebiges, aber festes \(n \in
	\mathbb{N}\). \\
	\(n \rightarrow n+1\): Sei \(B_{n+1} = A[b_1,\ldots,b_{n+1}]\) und \(b_i\) 
	ganz über \(B_{i-1}\) für jedes \(i \in \underline{n+1}\). Nach
	Induktionsvoraussetzung wissen wir, dass \(B_n\) endlich erzeugter A-Modul
	und ganz über A ist. Außerdem ist \(b_{n+1}\) ganz über \(B_n\) und damit
	auch \(B_n[b_{n+1}] \cong B_{n+1}\) endlich erzeugter A-Modul und \(B_{n+1}\)
	ganz über A.
	\item Zu zeigen ist nach dem Unterringkriterium, dass für \(b, b' \in 
	\overline{A}_B\) auch \(bb', b-b' \in \overline{A}_B\) und \(1 \in
	\overline{A}_B\). \\
	Die \(1\) ist offensichtlich ganz über A, also gilt \(1 \in \overline{A}_B\).
	Es sind \(b, b'\) ganz in A, also auch b' ganz in \(A[b]\), also folgt mit
	(\ref{2.1.2 a}), dass alle Elemente aus \(A[b,b']\) ganz über A sind, also
	insbesondere \(bb'\) und \(b-b'\). Also ist \(\overline{A}_B\) ein Unterring
	von B.
	\item B ist ganz über C, also gilt für ein \(b \in B\), dass
	\(b^m + c_{m-1}b^{m-1} + \ldots + c_0 = 0\) mit \(m \geq 1, c_i \in C\).
	Da \(c_0, \ldots, c_{m-1}\) ganz sind in A, ist (\ref{2.1.2 a}) anwendbar
	und \(A[c_0, \ldots, c_{m-1}]\) ist endlich erzeugter A-Modul und ganz über A.
	Außerdem ist \(b\) ganz über \(A[c_0, \ldots, c_{m-1}]\) und mit nochmaliger
	Anwendung folgt, dass auch \(C':=A[c_0, \ldots, c_{m-1}, b]\) endlich erzeugter
	A-Modul und ganz über A ist. Also \(A[b] \subseteq C' \subseteq B\) und
	mit Lemma~\ref{2.1.1} folgt, dass b ganz ist über A.
	\item A ist ein Ring, also müssen wir zeigen, dass \(A* = A-\{0\}\) ist.
	Sei \(a \in A \subseteq B\). Dann existiert \(b \in B\) mit \(ab = 1\).
	\(b\) ist ganz in A, also existieren \(a_i \in A\) und \(m \geq 1\) mit
	\begin{align*} &b^m + a_{m-1}b^{m-1} + \ldots + a_0 = 0 \\
	\Leftrightarrow\quad &b^ma^{m-1} + a_{m-1}b^{m-1}a^{m-1} + \ldots + a_0a^{m-1} = 0 \\
	\Leftrightarrow\quad &b = -(a_{m-1}b^{m-1}a^{m-1} + \ldots + a_0a^{m-1}) \in A.
	\end{align*}
	Also ist A ein Körper.
	\end{enumerate}
	\end{proof}
	
	\begin{lemma}\label{tupelvergleich}
	Sei \(M \subseteq \mathbb{N}^n_0\) und \(N(\alpha) = \sum^{n-1}_{i=0}
	\alpha_{n-i}r^i\) für ein \(r \in \mathbb{N}\), das größer ist
	als jede Komponente jedes Elements aus M. Dann gilt für \(\alpha, \alpha'
	\in M_n\) und \(\alpha \neq \alpha'\), dass \(N(\alpha) \neq N(\alpha')\).
	\end{lemma}

	Eine k-Algebra ist im Folgenden immer eine kommutative, assoziative
	k-Algebra mit Eins.

	\begin{proof}[Beweis]
	Wir führen eine Induktion über \(n \in \mathbb{N}\).

	Sei \(n = 1\) und \(\alpha, \alpha' \in M_n\) mit \(\alpha \neq \alpha'\)
	und \(N(\alpha) = N(\alpha')\). Dann folgt
	\begin{align*} \sum^{n-1}_{i=0} \alpha_{n-i}r^i &= \sum^{n-1}_{i=0}
	\alpha'_{n-i}r^i \\ \Leftrightarrow \alpha_{1} &= \alpha'_{1}. \end{align*}
	Das ist ein Widerspruch, also \(N(\alpha) \neq N(\alpha')\).

	Sei \(n > 1\) mit \(\alpha \neq \alpha'\) und \(N(\alpha) = N(\alpha')\).
	Falls \(\alpha_n = \alpha'_n\), betrachten wir \(\beta = (\alpha_1,\ldots,
	\alpha_{n-1})\) und \(\beta' = (\alpha'_1,\ldots,\alpha'_{n-1})\).
	Sonst folgt
	\begin{align*} \sum^{n-1}_{i=1} \alpha_{n-i}r^i &= \sum^{n-1}_{i=0}
	\alpha'_{n-i}r^i \\
	\Leftrightarrow \alpha_0 + \sum^{n-1}_{i=1} \alpha_{n-i}r^i &=
	\alpha'_0 + \sum^{n-1}_{i=0} \alpha'_{n-i}r^i \\
	\Leftrightarrow \sum^{n-1}_{i=1} \alpha_{n-i}r^i - \sum^{n-1}_{i=1}
	\alpha'_{n-i}r^i &= \alpha'_0 - \alpha_0 \\
	\Leftrightarrow (\sum^{n-1}_{i=1} \alpha_{n-i}r^{i-1} - \sum^{n-1}_{i=1}
	\alpha'_{n-i}r^{i-1})r &= \alpha'_0 - \alpha_0
	\end{align*}
	Es ist \(r > |\alpha'_0 - \alpha_0| > 0\) nach Voraussetzung, aber
	\(r\;|\;\alpha'_0 - \alpha_0\). Also haben wir einen Widerspruch und damit
	folgt insgesamt per Induktion die Behauptung.
	\end{proof}

	\begin{satz}[Noetherscher Normalisierungssatz] \label{2.1.4}
	Sei A eine endlich erzeugte k-Algebra. Dann existieren algebraisch
	unabhängige Elemente \(a_1, \ldots, a_d \in A\), so dass A ganz ist über
	dem Teilring \(k[a_1, \ldots, a_d]\).
	\end{satz}

	\begin{proof}[Beweis]
	Da A eine endlich erzeugte k-Algebra ist, existieren \(a_1,\ldots,a_n\) mit
	\(A = k[a_1,\ldots,a_n]\). Wir führen nun eine Induktion
	über \(n \in \mathbb{N}\).

	Sei \(n=0\). Dann ist \(A=k\) und die Behauptung folgt.

	Sei nun \(n > 0\). Angenommen \(a_1,\ldots,a_n\) sind algebraisch unabhängig,
	dann ist A auch ganz über \(k[a_1,\ldots,a_n]\) und die Behauptung folgt.
	Wir nehmen also an, dass \(a_1,\ldots,a_n\) nicht algebraisch unabhängig sind.
	Dann existiert ein nichtkonstantes Polynom \(F \in  k[X_1,\ldots,X_n]\)	mit
	\(F(a_1,\ldots,a_n) = 0\).
	Im Folgenden zeigen wir, dass (ggf. nach Umnummerierung) \(a_n\) ganz über
	\(k[a_1,\ldots,a_{n-1}]\) ist, wir das Problem also auf \(a_1,\ldots,a_{n-1}\)
	zurückführen können.

	Da F nicht konstant ist, hat F ohne Beschränkung der Allgemeinheit (bzw.
	nach Umnummerierung) irgendwo die Variable \(X_n\). Außerdem ist
	\begin{displaymath} F = \sum_{\alpha \in \mathbb{N}^n_0} a_{\alpha}X^{\alpha}
	\text{ mit } a_{\alpha}\in k.\end{displaymath}
	Wir definieren \(N(\alpha) = \sum^{n-1}_{i=0} \alpha_{n-i}r^i\) für
	\(\alpha \in \mathbb{N}^n_0\). Dabei wählen wir ein \(r \in \mathbb{N}\),
	das größer ist als jede Komponente jedes \(\alpha \in \mathbb{N}^n_0\) aus F
	mit \(a_{\alpha} \neq 0\). Dann folgt mit Lemma~\ref{tupelvergleich}, dass
	\(N(\alpha) \neq N(\alpha')\) für \(\alpha, \alpha' \in \mathbb{N}^n_0\) und
	\(\alpha \neq \alpha'\). Setzen wir nun
	\( r_i := r^{n-i} \text{ und } Y_i := X_i - X_n^{r_i}
	\text{ für } i \in \underline{n-1}\). Dann gilt für ein Monom \(X^{\alpha}\),
	dass \begin{align*}
	X^{\alpha} &= X_1^{\alpha_1}\cdots X_{n}^{\alpha_n} \\
	&= (Y_1 + X_n^{r_1})^{\alpha_1} \cdots (Y_{n-1} + X_n^{r_{n-1}})^{\alpha_{n-1}}X_n^{\alpha_n} \\
	&= X_n^{r_1\alpha_1 + \ldots + r_{n-1}\alpha_{n-1} + \alpha_n} + \sum_{i=0}^{N-1} h_iX_n^i \\
	&= X_n^{N(\alpha)} + \sum_{i=0}^{N(\alpha)-1} h_iX_n^i
	\end{align*}
	mit \(h_i \in k[Y_1, \ldots, Y_{n-1}]\). Sei \(N = max\{N(\alpha)\;|\;a_{\alpha}
	\neq 0\}\), dann kann man F schreiben als
	\begin{displaymath} \tilde{F} = \lambda X_n^N + \sum_{i=0}^{N-1} h_iX_n^i.
	\end{displaymath}
	Setzen wir num \(y_i := a_i - a_n^{r_i}\) für \(i \in \underline{n-1}\).
	Dann ist \(R := k[y_1,\ldots,y_{n-1}] \subseteq A\) ein Teilring von A.
	Sei außerdem \(g := \tilde{F}(y_1,\ldots,y_{n-1},X_n) \in R[X_n]\).
	Es ist \(g \neq 0\) und \(g(a_n) = 0\). Also liefert \(\frac{1}{\lambda}g\)
	die ganze Abhängigkeit von \(a_n\) in R.

	Die Elemente \(y_1,\ldots,y_{n-1}\) sind ganz über R, also auch \(a_1,\ldots,
	a_{n-1}\), da \(a_i = y_i + a_n^{r_i}\) für \(i \in \underline{n-1}\).
	Also ist mit Korollar~\ref{2.1.2}(~\ref{2.1.2 a}) A ganz über R. Falls
	\(a_1,\ldots,a_{n-1}\) algebraisch ist, folgt die Behauptung direkt, sonst
	per Induktion.
	\end{proof}

	\begin{lemma}\label{hilfslemma}
	Sei A ein Körper, \(R = k[a_1, \ldots, a_n]\) ein Ring mit \(a_1, \ldots,
	a_n \in A\) algebraisch unabhängig in k und A ganz über R.
	Dann ist R ein Körper und \(d = 0\).
	\end{lemma}

	\begin{proof}[Beweis]
	Nach Korollar~\ref{2.1.2} (4) ist R ein Körper. Angenommen \(d > 0\).
	Da R ein Körper ist, existiert ein Element \(e_1 \in k\) mit \(e_1a_1 = 1\),
	also gilt \(e_1a_1 - 1 = 0\) und \(a_1\) ist nicht algebraisch unabhängig in k.
	Also folgt \(d = 0\).
	\end{proof}

	\begin{satz}[Schwache Form von Hilbert's Nullstellensatz]\label{schwach}
	Sei k algebraisch abgeschlossen. Dann sind die maximalen Ideale in
	\(k[X_1, \ldots, X_n]\) genau die Ideale der Form \((X_1-v_1,\ldots,
	X_n-v_n)\) mit \(v_i \in k\). Allgemeiner gilt, falls A eine beliebige
	k-Algebra ist, dass \(A/I \cong k\) für jedes maximale Ideal I in A.
	\end{satz}

	\begin{proof}[Beweis]
	Wir betrachten \(k[X_1, \ldots, X_n]\) und das Ideal \(I = (X_1 - v_i,
	\ldots, X_n - v_n)\). I ist maximal, weil \(k[X_1, \ldots, X_n]/I \cong k\).

	Andersrum sei I ein maximales Ideal und \(A := k[X_1, \ldots, X_n]/I\).
	Dann ist A ein Körper und also auch eine endlich erzeugte k-Algebra. Also
	existieren algebraisch unabhängige \(a_1, \ldots, a_d \in A\), so dass
	der Körper A nach Satz~\ref{2.1.4} ganz über \(R := k[a_1,\ldots, a_d]\) ist.
	Nach Lemma~\ref{hilfslemma} ist \(d = 0\).
	Also folgt \(R = k\) und \(A\) ist eine endliche algebraische Erweiterung von
	\(R\). Da k algebraisch abgeschlossen ist, folgt damit \(A = k\).
	Damit existieren für alle \(i \in \underline{n}\) \(v_i \in k\) mit
	\((X_i - v_i) \in I\) und \(I = (X_1 - v_1, \ldots, X_n - v_n)\).
	\end{proof}

	\begin{korollar}[Umformulierung der schwachen Form von Hilbert's Nullstellensatz]
	Sei k algebraisch abgeschlossen und I ein Ideal aus dem Polynomring \(R :=
	k[X_1,\ldots,X_n]\) mit \(I \neq R\). Dann ist die Varietät \(V(I)\) nicht leer.
	\end{korollar}

	\begin{proof}[Beweis]
	Nach Satz~\ref{schwach} existiert ein maximales Ideal M mit \(I \unlhd M \unlhd R\)
	und \(M = (X_1 - v_1, \ldots, X_n - v_n)\) für ein \(v \in k^n\). Also ist
	\(v \in V(M)\). Aus vorherigen Vorträgen wissen wir, dass aus \(I \subseteq M\) folgt,
	dass \(V(M) \subseteq V(I)\). Da \(v \in V(M)\), gilt also auch \(v \in V(I)\)
	und \(V(I) \neq \emptyset\).
	\end{proof}


\section{Normale Form}
\section{Starke Form}
\section{Anwendung}

\end{document}
