\documentclass{article}

\usepackage[utf8]{inputenc}
\usepackage{ngerman}
\usepackage{lmodern}
\usepackage{amsthm}
\usepackage{amssymb}
\usepackage{amsmath}
\usepackage[paper=a4paper,left=25mm,right=25mm,top=25mm]{geometry}
\usepackage{mathtools}
\usepackage{graphicx}
\usepackage{paralist}
\usepackage{color}

\newtheorem{satz}{Satz}
\newtheorem{definition}[satz]{Definition}
\newtheorem{lemma}[satz]{Lemma}
\newtheorem{proposition}[satz]{Proposition}

\title{Dimension von Varietäten}
\date{\today}
\author{Yvan Ngumeteh \and Emma Ahrens}

\begin{document}

\maketitle
%\tableofcontents

\section{Abstract}
\section{Einleitung}
\section{Hyperebenen}
\section{Schwache Form}

	Im Folgenden sind A und B kommmutative Ringe mit Eins und \(A \subseteq B\).

	\begin{definition}[Ganze Elemente]
	Wir nennen \(b \in B\) ganz über A, wenn es Elemente \(a_1, \ldots, a_n \in
	A\) gibt mit
	\begin{displaymath} b^n + a_{n-1}b^{n-1} + \cdots + a_1b + a_0 = 0
	\end{displaymath} für ein \(n \in \mathbb{N}\).
	Außerdem heißt B ganz über A, wenn jedes Element aus B ganz über A ist.
	\end{definition}

	\begin{lemma}
	Sei \(b \in B\). Dann ist äquivalent:
	\begin{enumerate}
	\item b ist ganz über A
	\item Der von b generierte Teilring \(A[b] \subseteq B\) ist ein endlich
	erzeugter A-Modul.
	\item Es existiert ein Teilring \(C \subseteq B\) mit \(A[b] \subseteq C\)
	und C ist ein endlich erzeugter A-Modul.
	\end{enumerate}
	\end{lemma}

	\begin{proof}[Beweis]
	(\(1 \Rightarrow 2\)): Es ist \(A[b] = \{f(b)\;|\;f\in A[X]\}\)
	und da b ganz ist, existiert ein Polynom \(0 \neq g \in A[X]\) mit \(g(b) 
	= 0\) und \(Grad(g) = n \geq 1\). Da \(A[X]\) ein euklidischer Ring ist, können
	wir jedes \(f \in A[X]\) schreiben als \(f = qg + r\) mit \(q,r \in A[X]\)
	und \(Grad(r) < n\). Also \(f(b) = q(b)*g(b) + r(b) = r(b)\) und f ist eine
	A-Linearkombination von \(1, b, b^2, \ldots, b^{n-1}\), also ist \(A[b]\)
	endlich generiert. \\
	(\(2 \Rightarrow 3\)): Setze \(C := A[b]\), dann ist C ein Teilring von B
	und die Aussage folgt. \\
	(\(3 \Rightarrow 1\)): Seien \(c_1, \ldots, c_n \in C\) mit \(C =
	\sum_{i=1}^n Ac_i\). 
	\end{proof}


\section{Normale Form}
\section{Starke Form}
\section{Anwendung}

\end{document}
