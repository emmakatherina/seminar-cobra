\documentclass{article}

\usepackage[utf8]{inputenc}
\usepackage{ngerman}
\usepackage{lmodern}
\usepackage{amsthm}
\usepackage{amssymb}
\usepackage{amsmath}
\usepackage[paper=a4paper,left=25mm,right=25mm,top=25mm]{geometry}
\usepackage{mathtools}
\usepackage{graphicx}
\usepackage{paralist}
\usepackage{color}

\newtheorem{satz}{Satz}
\newtheorem{definition}[satz]{Definition}
\newtheorem{lemma}[satz]{Lemma}
\newtheorem{proposition}[satz]{Proposition}
\newtheorem{korollar}[satz]{Korollar}
\newcommand*{\R}{k[X_{1},\ldots,X_{n}]}

\title{Hilbert's Nullstellensatz}
\date{\today}
\author{Yvan Ngumeteh \and Emma Ahrens}

\begin{document}

\maketitle
%\tableofcontents

\section{Abstract}

	Varietäten sind die Nullstellenmengen von Idealen aus einem Polynomring. Wir wollen hier die Beziehung zwischen der Menge der Varietäten und der Menge der Ideale charakterisieren und werden
	eine Bijektion zwischen der Menge der Varietäten und der Menge der Wurzelideale finden. 
	Dafür beweisen wir Hilbert's Nullstellensatz und schauen uns im Anschluss Anwendungen davon an.

\section{Einleitung}

	Sei \(S \subseteq \R\) eine Teilmenge eines Polynomrings über einem algebraisch abgeschlossenen
	Körper k. Dann ist \(V(S) := \{x \in k^n\,|\, f(x)=0\; \forall f \in S\}\) eine Varietät bzw.
	algebraische Menge. Andersherum nennt man für eine Teilmenge \(V \subseteq k^{n}\) das 
	Ideal \(I(V) := \{f \in \R\,|\, f(x)=0\; \forall x\in V\}\) das Verschwindungsideal von V.
	Hier definiert man die Abbildungen
	\begin{displaymath} \mathbf{I}: \{ V \subseteq k^n\,|\, V \text{algebraisch}\} \rightarrow \{ I \unlhd k[X_1, \ldots, X_n]\},\; V \mapsto \{f \in k[X_1, \ldots, X_n]\,|\, f(v) = 0 \; \forall v \in V\} \end{displaymath}
	und \begin{displaymath} \mathbf{V}: \{ I \unlhd k[X_1, \ldots, X_n]\} \rightarrow \{ V \subseteq k^n\,|\, V \text{algebraisch}\},\; I \mapsto \{v \in k\,|\, f(v) = 0 \; \forall f \in I\}. \end{displaymath}
	Nach \cite{Geck} sind die Abbildungen wohldefiniert und es gilt für \(S' \subseteq S \subseteq k[X_1,\ldots,X_n]\), dass \(\mathbf{V}(S) \subseteq \mathbf{V}(S')\) und für \(V' \subseteq V \subseteq k^n\), dass \(\mathbf{I}(V) \subseteq \mathbf{I}(V')\). \(\mathbf{V}\) und \(\mathbf{I}\) sind also inklusionsumkehrend und nach \cite{CLOS} ist \(\mathbf{V}\) surjektiv und \(\mathbf{I}\) injektiv.

	Betrachten wir beispielhaft \(I := \langle X^2 - 2X + 1\rangle \subseteq \mathbb{R}[X]\).
	Dann ist \(\mathbf{V}(I) = \{-1\} \subseteq \mathbb{R}\) die zugehörige Nullstellenmenge.
	Außerdem ist \(\sqrt{I} = \langle X-1\rangle \supseteq \langle (X-1)^2\rangle = I\) und
	\(\mathbf{V}(\sqrt{I}) = \{-1\}\), also
	\begin{displaymath}\mathbf{V}(I) = \mathbf{V}(\sqrt{I}).\end{displaymath}
	Andersrum ist \(\mathbf{I}(\{-1\}) = \sqrt{I}\), aber es existiert keine Teilmenge \(V \subseteq \mathbb{R}\) mit \(\mathbf{I}(V) = I\).
	
	Wir definieren uns nun das Wurzelideal eines Ideals \(I\) als kleinestes Ideal \(\sqrt{I} \subseteq k[X_1,\ldots,X_n]\) mit \(I \subseteq \sqrt{I}\) und für \(f^m \in \sqrt{I}\) für ein
	\(m \in \mathbb{N}\) folgt, dass \(f \in \sqrt{I}\). Im Folgenden werden wir zeigen, dass
	\(\mathbf{V}\) und \(\mathbf{I}\) bijektiv werden, wenn wir die Definitionsmenge bzw. die Bildmenge auf die Menge der Wurzelideale einschränken. Die beiden Abbildungen sind dann jeweils 
	invers zueinander und wir haben eine bijektive Beziehung zwischen der Menge der Wurzelideale und
	der Menge der entsprechenden Varietäten.
	
	Bevor wir uns die starke Form von Hilbert's Nullstellensatz anschauen, aus der die obige Bijektion folgt, beweisen wir jedoch erst die schwache und normale Form des Satzes.

\section{Schwache Form}
	%
	Wir beginnen zunächst mit der schwachen Form von Hilbert's Nullstellensatz,
	das heißt wir beweisen, dass die Varietät eines maximalen Ideals aus dem Polynomring \(k[X_1,\ldots,X_n]\) ein Punkt \(v \in k^n\) ist.
	Dafür wiederholen wir grundlegende Definitionen, beweisen eine Form des Noetherschen Normalisierungssatzes und folgern daraus dann schließlich die schwache Version von Hilbert's Nullstellensatz. 
	%
	
	%
	Fangen wir mit der Definiton von algebraisch unabhängigen Elementen in einer
	k-Algebra an, welche wir für den Noetherschen Normalisierungssatz brauchen werden.
	%
	
	\begin{definition}[Algebraische Elemente]
	Sei A eine k-Algebra. Dann heißt die Menge \(a_1,\ldots, a_m \in A\)
	algebraisch unabhängig, falls kein Polynom \(0 \neq F \in k[X_1,\ldots,X_m]
	\) existiert mit \(F(a_1,\ldots,a_m) = 0\).
	\end{definition}

	Sei zum Beispiel \(A:=\mathbb{Q}\). Dann sind \(\pi\) und \(e\) (die Eulersche Zahl) algebraisch unabhängig, da beide Elemente transzendent in \(\mathbb{Q}\) sind. Die Elemente \(\pi\) und \(\sqrt{2}\) sind jedoch algebraisch abhängig, da
	für \(f(X,Y) := Y^2 - 2 \neq 0\) gilt, dass \(f(\pi,\sqrt{2}) = 0\).
	
	Ebenfalls für den Noetherschen Normalisierungssatz benötigen wir die Definition von ganzen Elementen. Hierbei handelt es sich um eine Verallgemeinerung von algebraischen Elementen. \\

	Im Folgenden sind A und B kommmutative Ringe mit Eins und \(A \subseteq B\).

	\begin{definition}[Ganze Elemente]
	Wir nennen \(b \in B\) ganz über A, wenn es Elemente \(a_1, \ldots, a_n \in
	A\) gibt mit
	\begin{displaymath} b^n + a_{n-1}b^{n-1} + \cdots + a_1b + a_0 = 0
	\end{displaymath} für ein \(n \in \mathbb{N}\).
	Außerdem heißt B ganz über A, wenn jedes Element aus B ganz über A ist.
	\end{definition}

	Zum Beispiel sind \(\sqrt{2}\) und \(\sqrt{3}\) ganz über \(\mathbb{Q}\), \(\pi\) jedoch nicht. Es ist \(\mathbb{Q}[\sqrt{2}]\) eine Ringerweiterung von \(\mathbb{Q}\).
	
	Wir wollen nun kurz allgemeine Eigenschaften von Ringerweiterungen mit ganzen
	Elementen betrachten und beweisen dazu die folgenden Äquivalenzen.
		
	\begin{lemma} \label{2.1.1}
	Sei \(b \in B\). Dann ist äquivalent:
	\begin{enumerate}
	\item b ist ganz über A
	\item Der von b erzeugte Teilring \(A[b] \subseteq B\) ist ein endlich
	erzeugter A-Modul.
	\item Es existiert ein Teilring \(C \subseteq B\) mit \(A[b] \subseteq C\)
	und C ist ein endlich erzeugter A-Modul.
	\end{enumerate}
	\end{lemma}

	\begin{proof}[Beweis]
	(\(1 \Rightarrow 2\)): Es ist \(A[b] = \{f(b)\;|\;f\in A[X]\}\)
	und da b ganz ist, existiert ein Polynom \(0 \neq g \in A[X]\) mit \(g(b) 
	= 0\) und \(Grad(g) = n \geq 1\). Da \(A[X]\) ein euklidischer Ring ist, können
	wir jedes \(f \in A[X]\) schreiben als \(f = qg + r\) mit \(q,r \in A[X]\)
	und \(Grad(r) < n\). Also \(f(b) = q(b)*g(b) + r(b) = r(b)\) und f ist eine
	A-Linearkombination von \(1, b, b^2, \ldots, b^{n-1}\), also ist \(A[b]\)
	endlich erzeugt. \\
	(\(2 \Rightarrow 3\)): Setze \(C := A[b]\), dann ist C ein Teilring von B
	und die Aussage folgt. \\
	(\(3 \Rightarrow 1\)): Seien \(c_1, \ldots, c_n \in C\) mit \(C =
	\sum_{i=1}^n Ac_i\). Es gilt \(b \in A[b] \subseteq C\), also auch \(bc_i
	\in C\) und es existieren die \(a_{ij} \in A\) mit \(bc_i = \sum_{j=1}^n
	a_{ij}c_i\). Sei \(M \in A^{n\times n}\) eine Matrix mit \((M)_{i,j} =
	a_{ij}\) für alle \(i, j \in \underline{n}\) und \(v \in A^n\) der Vektor
	mit \(v_i = c_i\) wie oben. Dann entsprechen die obigen Gleichungen dem
	Gleichungssystem
	\begin{displaymath}Mv = bv \Leftrightarrow (M-I_n)v = 0.\end{displaymath}
	Die Cramersche Regel besagt, dass \(v_i = \frac{Det((M-I_n)_i)}{Det(M-I_n)}
	\Leftrightarrow v_iDet(M-I_n) = Det((M-I_n)_i)\), wobei in die Matrix
	\((M-I_n)_i \) in unserem Fall nur Nullen in der i-ten Spalte stehen. Also
	gilt \begin{displaymath} Det(M-I_n)_i) = 0 \Rightarrow v_iDet(M-I_n) = 0.
	\end{displaymath}
	Wir müssen noch zeigen, dass daraus \(Det(M-I_n) = 0\) folgt, denn dann 
	können wir die Determinante ausschreiben und \(1, b, \ldots\) wird linear
	abhängig über A, also ist b ganz über A. \\
	Es ist \(1 \in C\), also existiert eine Linearkombination \(1 =
	\sum_{i=1}^n a_ic_i \Leftrightarrow Det(M-I_n) = \sum_{i=1}^n
	a_ic_iDet(M-I_n) = 0\). Also gilt \(Det(M-I_n) = 0\) und die
	Behauptung folgt.
	\end{proof}
	
	Sei nun wieder \(A := \mathbb{Q}\) und \(b:=\sqrt{2}\). Dann ist \(\mathbb{Q}[\sqrt{2}]\) ganz über \(\mathbb{Q}\) nach dem vorherigen Lemma. Wir wissen,
	dass auch \(\sqrt{3}\) ganz über \(\mathbb{Q}\) ist. Wir zeigen gleich, dass
	dann auch \(\mathbb{Q}[\sqrt{2},\sqrt{3}]\) ganz über \(\mathbb{Q}\) ist.
	Außerdem definieren wir den ganzen Abschluss eines Ring in einem anderen Ring, beweisen, dass die Ganzheit von Ringen zueinander transitiv ist, und zeigen, dass wir für die Ringe \(A, B\) mit \(B\) ganz über \(A\) und \(B\) ein
	Körper folgern können, dass auch \(A\) ein Körper ist.
	
	Nachdem wir diese Folgerungen aus dem obigen Lemma gezeigt haben, beweisen wir noch ein sehr technisches, aber kurzes Lemma und können uns dann endlich dem 
	Noetherschen Normalisierungslemma widmen.

	\begin{korollar} \label{2.1.2}
	Seien A, B kommutative Ringe mit \(A \subseteq B\).
	\begin{enumerate}
	\item\label{2.1.2 a} Falls \(B = A[b_1, \ldots, b_n]\), wobei jedes \(b_i
	\in B\) ganz über \(A[b_1, \ldots, b_{i-1}]\) ist, dann ist B endlich
	erzeugter A-Modul und ganz über A.
	\item Die Menge \(\overline{A}_B := \{b \in B\;|\;b \text{ ganz über } A\}\) ist
	ein Teilring von B und heißt ganzer Abschluss von A in B.
	\item Sei \(C \subseteq B\) ein Teilring mit \(A \subseteq C\). Falls C
	ganz ist über A und B ganz ist über C, dann ist auch B ganz über A.
	\item Falls B ein Körper ist und ganz über A, dann ist A auch ein Körper.
	\end{enumerate}
	\end{korollar}

	\begin{proof}[Beweis]
	\begin{enumerate}
	\item Beweis durch Induktion über \(n \in \mathbb{N}\).\\
	\(n = 1\): Sei \(B_n = B_1 = A[b_1]\) und \(b_1\) ganz über A. Dann folgt
	mit Lemma~\ref{2.1.1}, dass \(A[b_1]\) ein endlich erzeugter A-Modul ist
	und \(A[b_1]\) ganz über A ist. \\
	Angenommen die Behauptung gilt für ein beliebiges, aber festes \(n \in
	\mathbb{N}\). \\
	\(n \rightarrow n+1\): Sei \(B_{n+1} = A[b_1,\ldots,b_{n+1}]\) und \(b_i\) 
	ganz über \(B_{i-1}\) für jedes \(i \in \underline{n+1}\). Nach
	Induktionsvoraussetzung wissen wir, dass \(B_n\) endlich erzeugter A-Modul
	und ganz über A ist. Außerdem ist \(b_{n+1}\) ganz über \(B_n\) und damit
	auch \(B_n[b_{n+1}] \cong B_{n+1}\) endlich erzeugter A-Modul und \(B_{n+1}\)
	ganz über A.
	\item Zu zeigen ist nach dem Unterringkriterium, dass für \(b, b' \in 
	\overline{A}_B\) auch \(bb', b-b' \in \overline{A}_B\) und \(1 \in
	\overline{A}_B\). \\
	Die \(1\) ist offensichtlich ganz über A, also gilt \(1 \in \overline{A}_B\).
	Es sind \(b, b'\) ganz in A, also auch b' ganz in \(A[b]\), also folgt mit
	(\ref{2.1.2 a}), dass alle Elemente aus \(A[b,b']\) ganz über A sind, also
	insbesondere \(bb'\) und \(b-b'\). Also ist \(\overline{A}_B\) ein Unterring
	von B.
	\item B ist ganz über C, also gilt für ein \(b \in B\), dass
	\(b^m + c_{m-1}b^{m-1} + \ldots + c_0 = 0\) mit \(m \geq 1, c_i \in C\).
	Da \(c_0, \ldots, c_{m-1}\) ganz sind in A, ist (\ref{2.1.2 a}) anwendbar
	und \(A[c_0, \ldots, c_{m-1}]\) ist endlich erzeugter A-Modul und ganz über A.
	Außerdem ist \(b\) ganz über \(A[c_0, \ldots, c_{m-1}]\) und mit nochmaliger
	Anwendung folgt, dass auch \(C':=A[c_0, \ldots, c_{m-1}, b]\) endlich erzeugter
	A-Modul und ganz über A ist. Also \(A[b] \subseteq C' \subseteq B\) und
	mit Lemma~\ref{2.1.1} folgt, dass b ganz ist über A.
	\item A ist ein Ring, also müssen wir zeigen, dass \(A* = A-\{0\}\) ist.
	Sei \(a \in A \subseteq B\). Dann existiert \(b \in B\) mit \(ab = 1\).
	\(b\) ist ganz in A, also existieren \(a_i \in A\) und \(m \geq 1\) mit
	\begin{align*} &b^m + a_{m-1}b^{m-1} + \ldots + a_0 = 0 \\
	\Leftrightarrow\quad &b^ma^{m-1} + a_{m-1}b^{m-1}a^{m-1} + \ldots + a_0a^{m-1} = 0 \\
	\Leftrightarrow\quad &b = -(a_{m-1}b^{m-1}a^{m-1} + \ldots + a_0a^{m-1}) \in A.
	\end{align*}
	Also ist A ein Körper.
	\end{enumerate}
	\end{proof}
	
	Im Beweis des Noetherschen Normalisierungssatzes betrachten wir Tupel,
	welche auf natürliche Zahlen abgebildet werden. Nun wollen wir aus der Gleichheit der natürlichen Zahlen folgern können, dass auch die Tupel gleich sind. Dafür benutzen wir das folgende Lemma. (Die erwähnte Abbildung ist \(N\).) 
	
	\begin{lemma}\label{tupelvergleich}
	Sei \(M \subseteq \mathbb{N}^n_0\) und \(N(\alpha) = \sum^{n-1}_{i=0}
	\alpha_{n-i}r^i\) für ein \(r \in \mathbb{N}\), das größer ist
	als jede Komponente jedes Elements aus M. Dann gilt für \(\alpha, \alpha'
	\in M_n\) und \(\alpha \neq \alpha'\), dass \(N(\alpha) \neq N(\alpha')\).
	\end{lemma}

	Eine k-Algebra ist im Folgenden immer eine kommutative, assoziative
	k-Algebra mit Eins.

	\begin{proof}[Beweis]
	Wir führen eine Induktion über \(n \in \mathbb{N}\).

	Sei \(n = 1\) und \(\alpha, \alpha' \in M_n\) mit \(\alpha \neq \alpha'\)
	und \(N(\alpha) = N(\alpha')\). Dann folgt
	\begin{align*} \sum^{n-1}_{i=0} \alpha_{n-i}r^i &= \sum^{n-1}_{i=0}
	\alpha'_{n-i}r^i \\ \Leftrightarrow \alpha_{1} &= \alpha'_{1}. \end{align*}
	Das ist ein Widerspruch, also \(N(\alpha) \neq N(\alpha')\).

	Sei \(n > 1\) mit \(\alpha \neq \alpha'\) und \(N(\alpha) = N(\alpha')\).
	Falls \(\alpha_n = \alpha'_n\), betrachten wir \(\beta = (\alpha_1,\ldots,
	\alpha_{n-1})\) und \(\beta' = (\alpha'_1,\ldots,\alpha'_{n-1})\).
	Sonst folgt
	\begin{align*} \sum^{n-1}_{i=1} \alpha_{n-i}r^i &= \sum^{n-1}_{i=0}
	\alpha'_{n-i}r^i \\
	\Leftrightarrow \alpha_0 + \sum^{n-1}_{i=1} \alpha_{n-i}r^i &=
	\alpha'_0 + \sum^{n-1}_{i=0} \alpha'_{n-i}r^i \\
	\Leftrightarrow \sum^{n-1}_{i=1} \alpha_{n-i}r^i - \sum^{n-1}_{i=1}
	\alpha'_{n-i}r^i &= \alpha'_0 - \alpha_0 \\
	\Leftrightarrow (\sum^{n-1}_{i=1} \alpha_{n-i}r^{i-1} - \sum^{n-1}_{i=1}
	\alpha'_{n-i}r^{i-1})r &= \alpha'_0 - \alpha_0
	\end{align*}
	Es ist \(r > |\alpha'_0 - \alpha_0| > 0\) nach Voraussetzung, aber
	\(r\;|\;\alpha'_0 - \alpha_0\). Also haben wir einen Widerspruch und damit
	folgt insgesamt per Induktion die Behauptung.
	\end{proof}

	Damit kommen wir nun endlich zu dem Satz mit dessen Hilfe wir die schwache 
	Form von Hilbert's Nullstellensatz beweisen werden. Wir zeigen also, dass 
	wir für jede endlich erzeugte k-Algebra \(A\) einen Teilring finden, über 
	dem \(A\) ganz ist.
	
	\begin{satz}[Noetherscher Normalisierungssatz] \label{2.1.4}
	Sei A eine endlich erzeugte k-Algebra. Dann existieren algebraisch
	unabhängige Elemente \(a_1, \ldots, a_d \in A\), so dass A ganz ist über
	dem Teilring \(k[a_1, \ldots, a_d]\).
	\end{satz}

	\begin{proof}[Beweis]
	Da A eine endlich erzeugte k-Algebra ist, existieren \(a_1,\ldots,a_n\) mit
	\(A = k[a_1,\ldots,a_n]\). Wir führen nun eine Induktion
	über \(n \in \mathbb{N}\).

	Sei \(n=0\). Dann ist \(A=k\) und die Behauptung folgt.

	Sei nun \(n > 0\). Angenommen \(a_1,\ldots,a_n\) sind algebraisch unabhängig,
	dann ist A auch ganz über \(k[a_1,\ldots,a_n]\) und die Behauptung folgt.
	Wir nehmen also an, dass \(a_1,\ldots,a_n\) nicht algebraisch unabhängig sind.
	Dann existiert ein nichtkonstantes Polynom \(F \in  k[X_1,\ldots,X_n]\)	mit
	\(F(a_1,\ldots,a_n) = 0\).
	Im Folgenden zeigen wir, dass (ggf. nach Umnummerierung) \(a_n\) ganz über
	\(k[a_1,\ldots,a_{n-1}]\) ist, wir das Problem also auf \(a_1,\ldots,a_{n-1}\)
	zurückführen können.

	Da F nicht konstant ist, hat F ohne Beschränkung der Allgemeinheit (bzw.
	nach Umnummerierung) irgendwo die Variable \(X_n\). Außerdem ist
	\begin{displaymath} F = \sum_{\alpha \in \mathbb{N}^n_0} a_{\alpha}X^{\alpha}
	\text{ mit } a_{\alpha}\in k.\end{displaymath}
	Wir definieren \(N(\alpha) = \sum^{n-1}_{i=0} \alpha_{n-i}r^i\) für
	\(\alpha \in \mathbb{N}^n_0\). Dabei wählen wir ein \(r \in \mathbb{N}\),
	das größer ist als jede Komponente jedes \(\alpha \in \mathbb{N}^n_0\) aus F
	mit \(a_{\alpha} \neq 0\). Dann folgt mit Lemma~\ref{tupelvergleich}, dass
	\(N(\alpha) \neq N(\alpha')\) für \(\alpha, \alpha' \in \mathbb{N}^n_0\) und
	\(\alpha \neq \alpha'\). Setzen wir nun
	\( r_i := r^{n-i} \text{ und } Y_i := X_i - X_n^{r_i}
	\text{ für } i \in \underline{n-1}\). Dann gilt für ein Monom \(X^{\alpha}\),
	dass \begin{align*}
	X^{\alpha} &= X_1^{\alpha_1}\cdots X_{n}^{\alpha_n} \\
	&= (Y_1 + X_n^{r_1})^{\alpha_1} \cdots (Y_{n-1} + X_n^{r_{n-1}})^{\alpha_{n-1}}X_n^{\alpha_n} \\
	&= X_n^{r_1\alpha_1 + \ldots + r_{n-1}\alpha_{n-1} + \alpha_n} + \sum_{i=0}^{N-1} h_iX_n^i \\
	&= X_n^{N(\alpha)} + \sum_{i=0}^{N(\alpha)-1} h_iX_n^i
	\end{align*}
	mit \(h_i \in k[Y_1, \ldots, Y_{n-1}]\). Sei \(N = max\{N(\alpha)\;|\;a_{\alpha}
	\neq 0\}\), dann kann man F schreiben als
	\begin{displaymath} \tilde{F} = \lambda X_n^N + \sum_{i=0}^{N-1} h_iX_n^i.
	\end{displaymath}
	Setzen wir nun \(y_i := a_i - a_n^{r_i}\) für \(i \in \underline{n-1}\).
	Dann ist \(R := k[y_1,\ldots,y_{n-1}] \subseteq A\) ein Teilring von A.
	Sei außerdem \(g := \tilde{F}(y_1,\ldots,y_{n-1},X_n) \in R[X_n]\).
	Es ist \(g \neq 0\) und \(g(a_n) = 0\). Also liefert \(\frac{1}{\lambda}g\)
	die ganze Abhängigkeit von \(a_n\) in R.

	Die Elemente \(y_1,\ldots,y_{n-1}\) sind ganz über R, also auch \(a_1,\ldots,
	a_{n-1}\), da \(a_i = y_i + a_n^{r_i}\) für \(i \in \underline{n-1}\).
	Also ist mit Korollar~\ref{2.1.2}(~\ref{2.1.2 a}) A ganz über R. Falls
	\(a_1,\ldots,a_{n-1}\) algebraisch ist, folgt die Behauptung direkt, sonst
	per Induktion.
	\end{proof}
	
	Damit der Beweis zur schwachen Form von Hilbert's Nullstellensatz nicht
	zu lang wird, betrachten wir hier noch ein kleines Hilfslemma, welches das
	Ergebnis aus Korollar~\ref{2.1.2}(4) mit dem Noetherschen Normalisierungslemma verbinden.

	\begin{lemma}\label{hilfslemma}
	Sei A ein Körper, \(R = k[a_1, \ldots, a_n]\) ein Ring mit \(a_1, \ldots,
	a_n \in A\) algebraisch unabhängig in k und A ganz über R.
	Dann ist R ein Körper und damit \(n = 0\).
	\end{lemma}

	\begin{proof}[Beweis]
	Nach Korollar~\ref{2.1.2} (4) ist R ein Körper. Angenommen \(n > 0\).
	Da R ein Körper ist, existiert ein Element \(e_1 \in k\) mit \(e_1a_1 = 1\),
	also gilt \(e_1a_1 - 1 = 0\) und \(a_1\) ist nicht algebraisch unabhängig in k.
	Also folgt \(n = 0\).
	\end{proof}
	
	Nun können wir endlich die schwache Form von Hilbert's Nullstellensatz beweisen, welche uns
	zum Beispiel sagt, dass die Varietät eines jeden Ideals \(I \subsetneq k[X_1,\ldots,X_n]\) nicht leer ist.

	\begin{satz}[Schwache Form von Hilbert's Nullstellensatz]\label{schwach}
	Sei k algebraisch abgeschlossen. Dann sind die maximalen Ideale in
	\(k[X_1, \ldots, X_n]\) genau die Ideale der Form \((X_1-v_1,\ldots,
	X_n-v_n)\) mit \(v_i \in k\). Allgemeiner gilt, falls A eine beliebige
	k-Algebra ist, dass \(A/I \cong k\) für jedes maximale Ideal I in A.
	\end{satz}

	\begin{proof}[Beweis]
	Wir betrachten \(k[X_1, \ldots, X_n]\) und das Ideal \(I = (X_1 - v_i,
	\ldots, X_n - v_n)\). I ist maximal, weil \(k[X_1, \ldots, X_n]/I \cong k\).

	Andersrum sei I ein maximales Ideal und \(A := k[X_1, \ldots, X_n]/I\).
	Dann ist A ein Körper und also auch eine endlich erzeugte k-Algebra. Also
	existieren algebraisch unabhängige \(a_1, \ldots, a_d \in A\), so dass
	der Körper A nach Satz~\ref{2.1.4} ganz über \(R := k[a_1,\ldots, a_d]\) ist.
	Nach Lemma~\ref{hilfslemma} ist \(d = 0\).
	Also folgt \(R = k\) und \(A\) ist eine endliche algebraische Erweiterung von
	\(R\). Da k algebraisch abgeschlossen ist, folgt damit \(A = k\).
	Damit existieren für alle \(i \in \underline{n}\) \(v_i \in k\) mit
	\((X_i - v_i) \in I\) und \(I = (X_1 - v_1, \ldots, X_n - v_n)\).
	\end{proof}
	
	Wir wissen jetzt also, dass die Varietät eines maximalen Ideals \(I \subseteq 		k[X_1,\ldots,X_n]\) ein Punkt \(v \in k^n\) ist. Daraus können wir jetzt direkt noch folgern, dass die Varietät jedes echten Ideals des Polynomrings nicht-leer ist.

	\begin{korollar}[Umformulierung der schwachen Form von Hilbert's Nullstellensatz]
	Sei k algebraisch abgeschlossen und I ein Ideal aus dem Polynomring \(R :=
	k[X_1,\ldots,X_n]\) mit \(I \neq R\). Dann ist die Varietät \(V(I)\) nicht leer.
	\end{korollar}

	\begin{proof}[Beweis]
	Nach Satz~\ref{schwach} existiert ein maximales Ideal M mit \(I \unlhd M \unlhd R\)
	und \(M = (X_1 - v_1, \ldots, X_n - v_n)\) für ein \(v \in k^n\). Also ist
	\(v \in V(M)\). Aus vorherigen Vorträgen wissen wir, dass aus \(I \subseteq M\) folgt,
	dass \(V(M) \subseteq V(I)\). Da \(v \in V(M)\), gilt also auch \(v \in V(I)\)
	und \(V(I) \neq \emptyset\).
	\end{proof}

	Hiermit haben wir die schwache Form von Hilbert's Nullstellensatz bewiesen. Damit können wir jetzt relativ einfach die normale und starke Form beweisen und folgern schließlich, dass eine Bijektion zwischen den Wurzelidealen eines Polynomrings und den zugehörigen Varietäten existiert.

\section{Normale Form}
\section{Starke Form}
\section{Anwendung}

\begin{thebibliography}{99}
	\bibitem{CLOS}
	\textsc{Cox}, David; \textsc{Little}, John; \textsc{O'Shea}, Donal:
	\newblock \emph{Ideals, Varieties, and Algorithms}.
	\newblock Third Edition
	\newblock Springer-Verlag, 2007
	
	\bibitem{Geck}
	\textsc{Geck}, Meinolf:
	\newblock \emph{An introduction to algebraic geometry and algebraic groups}
	\newblock Clarendon Press - Oxford, 2003
\end{thebibliography}


\end{document}
