\documentclass{article}

\usepackage[utf8]{inputenc}
\usepackage{ngerman}
\usepackage{lmodern}
\usepackage{amsthm}
\usepackage{amssymb}
\usepackage{amsmath}
\usepackage[paper=a4paper,left=25mm,right=25mm,top=25mm]{geometry}
\usepackage{mathtools}
\usepackage{graphicx}
\usepackage{paralist}
\usepackage{color}

\newtheorem{satz}{Satz}
\newtheorem{definition}[satz]{Definition}
\newtheorem{lemma}[satz]{Lemma}
\newtheorem{proposition}[satz]{Proposition}
\newtheorem{korollar}[satz]{Korollar}

\title{Hilbert's Nullstellensatz}
\date{\today}
\author{Yvan Ngumeteh \and Emma Ahrens}

\begin{document}

\maketitle
	
	\begin{satz}[Hilbert's Nullstellensatz für Hyperebenen]
	Sei k algebraisch abgeschlossen, \(f \in k[X_1,\ldots,X_n]\) nicht
	konstant und \(\emptyset \neq H_f \subseteq k^n\) die korrespondierende
	Hyperebene. Wir können f schreiben als \(f = f_1^{n_1}\cdots f_r^{n_r}\)
	mit \(f_1,\ldots,f_r\) irreduzibel und paarweise teilerfremd. Dann ist
	\begin{displaymath} H_f = H_{f_1} \cup \cdots \cup H_{f_r} \text{und }
	\mathbf{I}(H_f) = (f_1\cdots f_r). \end{displaymath}
	Insbesondere gilt, falls f irreduzibel ist, dass \(\mathbf{I}(H_f) = (f)\).
	\end{satz}

	\begin{definition}[Algebraische Elemente]
	Sei A eine k-Algebra. Dann heißt die Menge \(a_1,\ldots, a_m \in A\)
	algebraisch unabhängig, falls kein Polynom \(0 \neq F \in k[X_1,\ldots,X_m]
	\) existiert mit \(F(a_1,\ldots,a_m) = 0\).
	\end{definition}

	Im Folgenden sind A und B kommmutative Ringe mit Eins und \(A \subseteq B\).

	\begin{definition}[Ganze Elemente]
	Wir nennen \(b \in B\) ganz über A, wenn es Elemente \(a_1, \ldots, a_n \in
	A\) gibt mit
	\begin{displaymath} b^n + a_{n-1}b^{n-1} + \cdots + a_1b + a_0 = 0
	\end{displaymath} für ein \(n \in \mathbb{N}\).
	Außerdem heißt B ganz über A, wenn jedes Element aus B ganz über A ist.
	\end{definition}

	\begin{lemma}
	Sei \(b \in B\). Dann ist äquivalent:
	\begin{enumerate}
	\item b ist ganz über A
	\item Der von b erzeugte Teilring \(A[b] \subseteq B\) ist ein endlich
	erzeugter A-Modul.
	\item Es existiert ein Teilring \(C \subseteq B\) mit \(A[b] \subseteq C\)
	und C ist ein endlich erzeugter A-Modul.
	\end{enumerate}
	\end{lemma}

	\begin{korollar}
	Seien A, B kommutative Ringe mit \(A \subseteq B\).
	\begin{enumerate}
	\item Falls \(B = A[b_1, \ldots, b_n]\), wobei jedes \(b_i \in B\) ganz über
	\(A[b_1, \ldots, b_{i-1}]\) ist, dann ist B endlich erzeugter A-Modul und 
	ganz über A.
	\item Die Menge \(\bar{A}_B := \{b \in B\;|\;b \text{ ganz über } A\}\) ist
	ein Teilring von B und heißt ganzer Abschluss von A in B.
	\item Sei \(C \subseteq B\) ein Teilring mit \(A \subseteq C\). Falls C
	ganz ist über A und B ganz ist über C, dann ist auch B ganz über A.
	\item Falls B ein Körper ist und ganz über A, dann ist A auch ein Körper.
	\end{enumerate}
	\end{korollar}

	Eine k-Algebra ist im Folgenden immer eine kommutative, assoziative
	k-Algebra mit Eins.

	\begin{satz}[Noetherscher Normalisierungssatz]
	Sei A eine endlich erzeugte k-Algebra. Dann existieren algebraisch
	unabhängige Elemente \(a_1, \ldots, a_d \in A\), so dass A ganz ist über
	dem Teilring \(k[a_1, \ldots, a_d]\).
	\end{satz}

	\begin{satz}[Schwache Form von Hilbert's Nullstellensatz]
	Sei k algebraisch abgeschlossen. Dann sind die maximalen Ideale in
	\(k[X_1, \ldots, X_n]\) genau die Ideale der Form \((X_1-v_1,\ldots,
	X_n-v_n)\) mit \(v_i \in k\). Allgemeiner gilt, falls A eine beliebige
	k-Algebra ist, dass \(A/I \cong k\) für jedes maximale Ideal I in A.
	\end{satz}

\end{document}
